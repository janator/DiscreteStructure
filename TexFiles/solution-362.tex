\begin{task}{362}
	В выпуклом многограннике все грани являются правильными 5-, 6- или 7-угольниками. Докажите, что пятиугольных граней ровно на 12 больше, чем семиугольных.
\end{task}

\begin{solution} 
	Выпуклый многогранник можно уложить на сфере. Это эквивалентно тому, что граф планарен. Тогда для него верна формула Эйлера: \[ n - e + f = 2.\] Удвоенное количество ребер можно выразить через количество 5-, 6- и 7-угольников (каждое ребро входит в два многоугольника): \[ 2 e = 5 n_5 + 6 n_6 + 7 n_7.\] Количество граней многогранника тоже выражается через количество многоугольников: \[ f = n_5 + n_6 + n_7.\] В каждую вершину входит хотя бы три ребра, каждое ребро отвечает двум вершинам: \[ 2 e \geq 3 n.\] Так как у нас правильные 5-, 6- и 7-угольники, то можем ограничить число ребер и сверху. Минимальный угол во всех гранях -- $\dfrac{3 \pi}{5}$. Тогда в каждую вершину может войти не больше $ \dfrac{10 \pi}{3 \pi} = 3$ ребер. Значит: \[ 2 e = 3 n.\] Соберем все формулы в одну: 
	\begin{align*}
	    6n - 6e + 6f &= 12, \\
	    4e - 6e + 6f &= 12, \\ 
	    6n_5 + 6n_6 + 6n_7 - 5n_5 - 6n_6 - 7n_7 &= 12, \\
	    n_5 - n_7 &= 12.
	\end{align*}
\end{solution}