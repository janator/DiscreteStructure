\begin{task}{5}
	Планарен ли следующий граф? Если да, то нарисуйте его без самопересечений, если нет, то обоснуйте непланарность, найдя в нём подграф, гомеоморфный K3,3, и сославшись на теорему Куратовского.  \\
	\begin{figure}[h]
        \centering
        \includegraphics[width=0.5\linewidth]{img/66-1.png}
    \end{figure}
\end{task}

\begin{solution}
	Уберем часть ребер из графа.
		\begin{figure}[h]
        \centering
        \includegraphics[width=0.5\linewidth]{img/66-2.png}
    \end{figure}
    
    Исключим белые вершины, заменим просто ребрами, и получим граф $K_{3,3}$.
    	\begin{figure}[h]
        \centering
        \includegraphics[width=0.5\linewidth]{img/66-3.png}
    \end{figure}
    
    В исходном графе есть граф, гомеоморфный $K_{3,3}$. Значит, по теореме Куратовского граф не планарен.
\end{solution}