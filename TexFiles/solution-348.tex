\begin{task}{348}
	Пусть $\mathbb {G}$ — произвольная группа, и $\mathbb{H \trianglelefteq G}$ — её нормальная подгруппа. Докажите, что $\mathbb{H}$ является объединением некоторых классов сопряжённости в $\mathbb{G}$. Докажите также, что если взять объединение всех классов сопряженности, состоящих ровно из одного элемента, то получившееся множество будет являться нормальной подгруппой в $\mathbb{G}$.
\end{task}

\begin{solution} 
Запишем оба определения.
    \begin{enumerate}
	\item Подгруппа $\mathbb{H}$ группы $\mathbb{G}$ является нормальной, если для любого элемента $h \in \mathbb{H}$ и любого $g \in \mathbb{G}$ элемент $ghg^{-1}$ лежит в $\mathbb{H}$. 
	\item Элементы $g_1$ и $g_2$ группы $\mathbb{G}$ называются сопряженными, если существует элемент $h \in \mathbb{G}$, для которого $hg_1h^{-1}=g_2$. Класс сопряженности --- множество элементов группы $\mathbb{G}$, сопряженных заданному $g \in \mathbb{G}$.
	\end{enumerate}
	Тогда для любого элемента $h \in \mathbb{H}$ все его сопряженные элементы также лежат в $\mathbb{H}$. Значит, $\mathbb{H}$ является объединением классов сопряженности по всем $h \in \mathbb{H}$. Теперь рассмотрим класс сопряженности [$h_k$], состоящий из одного элемента:
	\begin{enumerate}
	    \item $\exists g \in \mathbb{G}:  g h_k g^{-1} = h_k$
	    \item $\forall x \in \mathbb{G} \ \nexists g \in \mathbb{G}: g h_k g^{-1} = x$ или $\forall g \in \mathbb{G}: g h_k g^{-1} = h_k$
	\end{enumerate}
	Видно, что второе условие удовлетворяет определению нормальной подгруппы и в $\bigcup$[$h_k$] будут лежать только сами эти элементы.
\end{solution}