\documentclass[12pt,a4paper]{extreport}
\usepackage[l2tabu,orthodox]{nag}

% Пожалуйста, не меняйте указанные ниже установки полей в документе
\usepackage[left=10mm,right=50mm, top=15mm,bottom=15mm,bindingoffset=0cm]{geometry}

\usepackage{indentfirst}

\usepackage{ccaption}
\captiondelim{. }

\usepackage{amssymb,amsmath,amsthm}

\usepackage{fontspec}

\setmainfont[Ligatures=TeX]{STIX}
\newfontfamily{\cyrillicfont}[Ligatures=TeX]{STIX}
\setmonofont{Fira Mono}
\newfontfamily{\cyrillicfonttt}{Fira Mono}

\usepackage{polyglossia}
\setdefaultlanguage{russian}
\setotherlanguage{english}


\usepackage{graphicx}
\graphicspath{{img/}}
\DeclareGraphicsExtensions{.pdf,.png,.jpg}


\usepackage{color}
\definecolor{darkblue}{rgb}{0,0,.75}
\definecolor{darkred}{rgb}{.7,0,0}
\definecolor{darkgreen}{rgb}{0,.7,0}

\usepackage[normalem]{ulem}
\usepackage[textwidth=4cm,textsize=tiny]{todonotes}
\newcommand{\fix}[2]{{\textcolor{red}{\uwave{#1}}\todo[fancyline]{#2}}}
\newcommand{\hl}[1]{{\textcolor{red}{#1}}}
\newcommand{\cmd}[1]{{\ttfamily{\textbackslash #1}}}

\newcommand{\vrb}[1]{\PVerb{#1}}
\newcommand{\vrbb}[1]{\texttt{\textbackslash}\PVerb{#1}}
\newcommand{\rules}{{\href{https://goo.gl/FhyJJm}{Правила}}}


\usepackage[
    draft = false,
    unicode = true,
    colorlinks = true,
    allcolors = blue,
    hyperfootnotes = true
]{hyperref}


\theoremstyle{plain}
\newtheorem{theorem}{Теорема}
\newtheorem{lemma}{Лемма}
\newtheorem{proposition}{Утверждение}
\newtheorem{corollary}{Следствие}
\theoremstyle{definition}
\newtheorem{definition}{Определение}
\newtheorem{notation}{Обозначение}
\newtheorem{example}{Пример}
    
\newenvironment{task}[1]
    {\par\noindent\textbf{\href{https://progensys.dainiak.com/problem-#1}{Задача #1}. }}
    {\smallskip}
\newenvironment{solution}
    {\par\noindent\textbf{Решение. }}
    {\bigskip}
    


\title{Решения задач по курсу «Дискретные структуры»}
\author{Виктория Рудых}

\begin{document}
\maketitle

% Таким образом нужно добавлять решения задач:
% Каждое решение — в отдельном файле с названием “solution-N.tex”, где N — номер задачи, включающий буквенное обозначение пункта задачи, если таковое имеется.
% Все картинки следует помещать в папку img/
% Все незаконченные тексты решений можно помещать в папку drafts
% В корневой папке должен быть только файл _main.tex и файлы вида solution-N.tex

%\input{solution-16.tex}
\begin{task}{5}
	Сформулируйте обратное утверждение к утверждению принципа Дирихле (для любой из формулировок принципа). Заметьте, что в любом случае можно сформулировать это обратное утверждение, хотя оно и будет неверным в общем случае для некоторых формулировок принципа Дирихле. 
\end{task}

\begin{solution}
	Приведем сначала формулировку принципа Дирихле, а потом сформулируем к ней обратную: <<Если количество кроликов больше количества клеток, то по крайней мере в одной клетке сидит больше одного кролика>>.  
	
	Обратное утверждение: <<Если хотя бы в одной клетке сидит больше одного кролика, то количество кроликов больше количества клеток>>.
\end{solution}
\begin{task}{37}
	Установите соответствие (не обязательно биекцию) между деревьями на рисунке и кодами Прюфера.  
	\begin{enumerate}
	    \item (5,4,8,9,2,1,4,1)
	    \item (5,4,8,9,2,1,1,4)
	    \item (5,4,8,9,2,4,1,1)
	\end{enumerate}
	
	\includegraphics[width=15cm]{img/1.png}
\end{task}

\begin{solution} \\
	Дерево a -- (5, 4, 8, 9, 2, 4, 1, 1) -- код 3;  \\
	Дерево b -- (5, 4, 8, 9, 2, 1, 4, 1) -- код 1;  \\
	Дерево c -- (5, 4, 1, 8, 9, 2, 1, 1) -- нет подходящего;
\end{solution}
\begin{task}{290}
    Сколькими способами можно рассадить за круглым девятиместным столом трёх англичан, трёх французов и трёх турок, так, чтобы никакая тройка соотечественников не сидела рядом (сидящие рядом пары соотечественников допустимы)? Рассадки, совмещающиеся поворотами круглого стола, считаются одинаковыми. 
\end{task}

\begin{solution}
	Пусть места пронумерованы. Тогда число вариантов рассадить людей за столом будет 9!. Посчитаем число способов, в которых три француза  сидят рядом. Будет девять вариантов выбрать место для них, еще 3! варианта рассадить их по этим трем соседним местам. Далее остаётся шесть мест, по которым можно рассадить оставшихся людей 6! способами. В итоге получаем \(9! \cdot 3! \cdot 6!\). Для англичан и турок способов будет столько же, но рассадки в которых сидят по трое рядом и французы и турки или и французы и англичане мы посчитали несколько раз. Пусть рядом сидят французы и англичане. Для французов выбрать три места всё также девять способов, из оставшихся шести мест выбрать три подряд для англичан можно четерьмя способами. И и тех и других между собой можно разместить 3! способами. Для французов и турок и турок и англичан симметрично. Теперь посчитаем сколько способов когда все сидят по трое. Выбрать для французов места девять способов, для турок два (слева или справа от французов, и во всех группах 3! вариантов рассадки между собой. В итоге:
	\[ 9! - 9 \cdot 3 \cdot 3! \cdot 6! + 9 \cdot 3 \cdot 3! \cdot 4 \cdot 3! \cdot 3! - 9 \cdot 3! \cdot 2 \cdot 3! \cdot 3! = 265680.\]
	Вспомним что повороты стола совпадают, так что:
	\[N = \frac{265680}{9} = 29520.\]
	Ответ: 29520.
\end{solution}
\begin{task}{11}
	На плоскости выбрано конечное количество точек, находящихся в общем положении (никакие три точки не лежат на одной прямой, никакие две не совпадают). Некоторые из выбранных точек соединены отрезками. Если два отрезка пересекаются, то их можно заменить двумя другими с концами в тех же точках (отрезки, имеющие лишь общий конец, не считаются пересекающимися). Может ли этот процесс продолжаться бесконечно? [Необходимо решить задачу непременно методом потенциалов, явно указав, какая функция используется в качестве «потенциала».]
\end{task}

\begin{solution} 
	Рассмотрим два пересекающихся отрезка. Зададим потенциал как сумму длин всех отрезков.  Четыре точки задают четырехугольник, сумма длин диагоналей в нём меньше сумм длин любых двух сторон. Тогда при замене пересекающихся отрезков на не пересекающиеся потенциал уменьшается. Так как у нас конечно число состояний системы, это значит что потенциал достигнет в какой-то момент минимума. Следовательно, процесс не может продолжаться бесконечно.
\end{solution}
\begin{task}{276}
	Перечислите все попарно неизоморфные связные неориентированные графы без петель и кратных рёбер с пятью вершинами и не более, чем шестью рёбрами.
\end{task}

\begin{solution} 
    Графы с четырьмя ребрами: \\
	\includegraphics[width=11cm]{img/276-4.png} \\
	Графы с пятью ребрами: \\
	\includegraphics[width=13cm]{img/276-3.png} \\
	Графы с шестью ребрами: \\
    \includegraphics[width=13cm]{img/276-2.png} 
\end{solution}
\begin{task}{362}
	В выпуклом многограннике все грани являются правильными 5-, 6- или 7-угольниками. Докажите, что пятиугольных граней ровно на 12 больше, чем семиугольных.
\end{task}

\begin{solution} 
	Выпуклый многогранник можно уложить на сфере. Это эквивалентно тому, что граф планарен. Тогда для него верна формула Эйлера: \[ n - e + f = 2.\] Удвоенное количество ребер можно выразить через количество 5-, 6- и 7-угольников (каждое ребро входит в два многоугольника): \[ 2 e = 5 n_5 + 6 n_6 + 7 n_7.\] Количество граней многогранника тоже выражается через количество многоугольников: \[ f = n_5 + n_6 + n_7.\] В каждую вершину входит хотя бы три ребра, каждое ребро отвечает двум вершинам: \[ 2 e \geq 3 n.\] Так как у нас правильные 5-, 6- и 7-угольники, то можем ограничить число ребер и сверху. Минимальный угол во всех гранях -- $\dfrac{3 \pi}{5}$. Тогда в каждую вершину может войти не больше $ \dfrac{10 \pi}{3 \pi} = 3$ ребер. Значит: \[ 2 e = 3 n.\] Соберем все формулы в одну: 
	\begin{align*}
	    6n - 6e + 6f &= 12, \\
	    4e - 6e + 6f &= 12, \\ 
	    6n_5 + 6n_6 + 6n_7 - 5n_5 - 6n_6 - 7n_7 &= 12, \\
	    n_5 - n_7 &= 12.
	\end{align*}
\end{solution}
\begin{task}{5}
	Планарен ли следующий граф? Если да, то нарисуйте его без самопересечений, если нет, то обоснуйте непланарность, найдя в нём подграф, гомеоморфный K3,3, и сославшись на теорему Куратовского.  \\
	\begin{figure}[h]
        \centering
        \includegraphics[width=0.5\linewidth]{img/66-1.png}
    \end{figure}
\end{task}

\begin{solution}
	Уберем часть ребер из графа.
		\begin{figure}[h]
        \centering
        \includegraphics[width=0.5\linewidth]{img/66-2.png}
    \end{figure}
    
    Исключим белые вершины, заменим просто ребрами, и получим граф $K_{3,3}$.
    	\begin{figure}[h]
        \centering
        \includegraphics[width=0.5\linewidth]{img/66-3.png}
    \end{figure}
    
    В исходном графе есть граф, гомеоморфный $K_{3,3}$. Значит, по теореме Куратовского граф не планарен.
\end{solution}
\begin{task}{91}
	Докажите, что если граф невозможно правильно раскрасить в $k - 1$ цвет, то для любой его правильной раскраски в $k$ цветов существует путь, в котором встречается ровно по одной вершине каждого цвета. 
\end{task}

\begin{solution} 
    Рассмотрим раскраску графа в $K$ цветов. Теперь рассмотрим вершины цвета два. Те из них, которые не связаны с вершинами цвета один, перекрасим в цвет один. Так как граф нельзя правильно раскрасить в $k - 1$ цвет, то осталась хотя бы одна вершина цвета два, которая связана с вершиной цвета один. Сделаем тоже самое для цвета три. Продолжим перекраску вплоть до цвета $k$. Найдем не перекрашенную вершину цвета $k$. Она будет связана с вершиной цвета $k - 1$ по построению. Та в свою очередь  $k - 2$, и так далее. Таким образом построен путь, вдоль которого встречаются вершины всех цветов.
\end{solution}
\begin{task}{294}
	Пусть граф $ G $ таков, что в нём степени вершин равны $ d_1 \leq d_2 \leq \ldots \leq d_n $ и для каждого $ i \in {1,2, \ldots ,⌊n/2⌋} $ выполнено хотя бы одно из двух неравенств $ d_i \geq i + 1 $ и $ d_{n−i} \geq n − i $. Докажите, что в $ G $ есть гамильтонов цикл.
\end{task}

\begin{solution} 
    Допустим, что данный граф негамильтонов. Тогда будем добавлять ребра в наш граф, пока не получим такой граф, что при добавлении любого ребра он становится гамильтоновым. Переупорядочим вершины чтобы соблюсти $ d_1 \leq d_2 \leq \ldots \leq d_n $ и для каждого $ i \in {1,2, \ldots ,⌊n/2⌋} $. Условие $\forall i \in {1,2, \ldots ,⌊n/2⌋} $ выполнено хотя бы одно из двух неравенств $ d_i \geq i + 1 $ и $ d_{n−i} \geq n − i $ сохранится при добавлении ребер. Теперь выберем пару таких несмежных вершин $A$ и $B$, что сумма их степеней максимальна. Между этими двумя вершинами есть гамильтонов путь по условию максимальности добавления ребер, иначе можно было бы добавить ребро между ними. Рассмотрим этот путь. Пусть $D$ самая правая вершина с которой связана $A$. Тогда левее неё в этой цепочке не может быть вершины $C$ с которой была бы связана вершина $B$, иначе был бы гамильтонов цикл $A \rightarrow D \rightarrow  \ldots \rightarrow B \rightarrow C \rightarrow \ldots \rightarrow A$.
    \includegraphics[width=13cm]{img/294-1.png} \\
    Из этого следует $d(B) \leq n - 1 - d(A)$ или $d(A) + d(B) < n$. Пусть степень $A$ меньше степени $B$. Вспомним, что $A$ и $B$ несвязные вершины с максимальной суммой степеней. Значит все вершины, с которыми $B$ не связана имеют степень меньше $d(A)$ и таких вершин будет $n - d(B) > d(x)$. То есть $d_1 < d(A), \ldots , d_{d(A)} \leq d(A) $. Следовательно $d_{n - d(A)} \geq n - d(A)$. Степени возрастают, так что  $d_{n - d(A)} \geq n - d(A), \ldots, d_n \geq n - d(A)$. Количество этих вершин больше чем степень $A$, значит среди них найдется одна вершина $C$ не связанная с ней, причем сумма их степеней $d(A) + d(C) \geq d(A) + n - d(A) = n > d(A) + d(B)$. Противоречие. Значит граф уже является гамильтоновым.
\end{solution}
\begin{task}{320}
	Пусть $e$~---~нейтральный элемент $S_n$. Найдите количество таких элементов $x \in S_n$, для которых $x^3 = e$. Ответ можно выписать в виде формулы со знаком суммирования $\sum$~---~и вряд ли получится по-другому.
\end{task}

\begin{solution} 
	Перестановка представима в виде композиции непересекающихся циклов. Рассмотрим цикл  $c$ длины $k$. Если $k = 1$, то тождественной перестановкой является только нейтральный элемент группы $e^3 = e$ (учтем его в ответе). При $k = 2$ два цикла уничтожатся $c \cdot c = e$, и тогда $c^3 = c$, что тоже не является тождественной перестановкой. При $k = 3$ все хорошо, $c^3 = e$. При больших $k$ очевидно тождественная перестановка не сможет получится. Теперь посчитаем количество элементов, представимых в виде композиции циклов длины три, $k$~---~количество непересекающихся циклов длины три. Если выбрано три элемента, то из них можно получить два различных цикла. Так как у нас циклы независимы, то их порядок не важен:
	\[1 + \sum_{k = 1}^{\lfloor\frac{n}{3}\rfloor} \dfrac{2^k}{k!} C_n^3 \cdot \ldots \cdot C_{n - 3(k - 1)}^3 = 1 +  \sum_{k = 1}^{\lfloor\frac{n}{3}\rfloor}\dfrac{2^k}{(3!)^k \cdot k!} \dfrac{n! \ldots (n - 3(k-1))!}{(n-3)! \ldots (n-3k)!} = 1 + \sum_{k = 1}^{\lfloor\frac{n}{3}\rfloor} \dfrac{n!}{3^k \cdot k! \cdot (n-3k)!}\]
\end{solution}
\begin{task}{142}
	Найдите асимптотику величины $\binom{3n+\log_2n}{2n}$ при $n \rightarrow +\infty$. Под нахождением асимптотики понимается нахождение такой формулы, между которой и оцениваемой величиной можно было бы поставить знак ∼. Формула внутри себя не должна содержать никаких значков $O$‑символики. Также не допускается наличие в формуле неопределённостей вида $\dfrac{0}{0}, 1^{\infty}$ и пр.
\end{task}

\begin{solution} 
    Воспользуемся формулой Стирлинга
    $ n! = \sqrt{2\pi n} \left( n\dfrac{n}{e}\right)$ и $ \log_2n = o(n)$ для получения асимптотики:
    \[
        \begin{split}
            \binom{3n+\log_2n}{2n} &=\dfrac{(3n+\log_2n)!} {(2n)! \cdot (n+\log_2n)!} \sim \dfrac{\sqrt{2\pi (3n+\log_2n)} \left( \dfrac{3n+\log_2n}{e}\right)^{3n+\log_2n}}{\sqrt{2\pi \cdot 2n} \left( \dfrac{2n}{e}\right)^{2n} \cdot \sqrt{2\pi (n+\log_2n)} \left( \dfrac{n+\log_2n}{e}\right)^{n+\log_2n}} = \\ &= \dfrac{\sqrt{3n+\log_2n} \cdot  (3n+\log_2n)^{3n+\log_2n}}{\sqrt{4\pi n} \cdot (2n)^{2n} \cdot \sqrt{n+\log_2n} \cdot (n+\log_2n)^{n+\log_2n}} = \\ &= \dfrac{\sqrt{3n} \cdot \left(1+\dfrac{\log_2n}{3n}\right)^{3n+\log_2n} \cdot (3n)^{3n+\log_2n}}{\sqrt{4\pi n} \cdot (2n)^{2n} \cdot \sqrt{n} \cdot \left(1+\dfrac{\log_2n}{n}\right)^{n+\log_2n} \cdot n^{n+\log_2n}} = \\ &= \sqrt{\dfrac{3}{4\pi n}} \cdot \dfrac{\exp{\left(\ln \left(1+\dfrac{\log_2n}{3n}\right)\cdot (3n+\log_2n)\right)} \cdot 3^{3n+\log_2n}}{ \exp{\left(\ln{\left(1+\dfrac{\log_2n}{n}\right)}\cdot (n+\log_2n)\right)} \cdot 2^{2n}} \sim \\ &\sim  \sqrt{\dfrac{3}{4\pi n}} \cdot \dfrac{\exp{\left( \left(\dfrac{\log_2n}{3n} - \dfrac{1}{2} \cdot \left(\dfrac{\log_2n}{3n}\right)^2 + o\left(\dfrac{\ln n}{n}\right)\right) \cdot (3n+\log_2n)\right)} \cdot 3^{3n+\log_2n}}{ \exp{\left( \left(\dfrac{\log_2n}{n} - \dfrac{1}{2} \cdot \left(\dfrac{\log_2n}{n}\right)^2 + o\left(\dfrac{\ln n}{n}\right)\right) \cdot (n+\log_2n)\right)} \cdot 2^{2n}} \sim \\ &\sim \sqrt{\dfrac{3}{4\pi n}} \cdot \dfrac{\exp{\left(\log_2n + \dfrac{1}{2} \cdot \dfrac{(\log_2n)^2}{3n}\right)} \cdot 3^{3n+\log_2n}}{ \exp{\left(\log_2n + \dfrac{1}{2} \cdot \dfrac{(\log_2n)^2}{n}\right)} \cdot 2^{2n}} \sim \\ &\sim \sqrt{\dfrac{3}{4\pi}} \cdot \dfrac{3^{3n+\log_2n}}{\sqrt{n} \cdot 2^{2n}} 
        \end{split}
    \]
\end{solution}
\begin{task}{159}
    Функция $n(s)$ задана в виде $n(s) := \max(n_0 | (n_0!)^{\ln n_0} \leq s)$. Найдите асимптотику этой функции при $s \rightarrow \infty$. 
\end{task}

\begin{solution} 
	Заметим, что $(n!)^{\ln(n+1)}.$ Возьмём логарифм от обеих частей:
	\[ \ln n \cdot \ln n! \leq \ln s \leq \ln (n+1) \cdot \ln (n+1)!.\]
	Воспользуясь формулой Стирлинга $\ln n! \sim n \cdot \ln n$ получим:
	\[ n \cdot \ln^2n \lesssim \ln s \lesssim (n+1) \cdot \ln^2(n+1).\]
	Учитывая, что $ (n+1) \cdot \ln^2(n+1) \sim n \cdot \ln^2n$, приходим к следующему выражению:
	\[ \ln s \sim n \cdot \ln^2 n.\]
	Возьмем от него еще раз логарифм $ \ln \ln s \sim \ln n$. Подставим это в предыдущее выражение: 
	\[n \sim \dfrac{\ln s}{(\ln \ln s)^2}\]
\end{solution}
\begin{task}{348}
	Пусть $\mathbb {G}$ — произвольная группа, и $\mathbb{H \trianglelefteq G}$ — её нормальная подгруппа. Докажите, что $\mathbb{H}$ является объединением некоторых классов сопряжённости в $\mathbb{G}$. Докажите также, что если взять объединение всех классов сопряженности, состоящих ровно из одного элемента, то получившееся множество будет являться нормальной подгруппой в $\mathbb{G}$.
\end{task}

\begin{solution} 
Запишем оба определения.
    \begin{enumerate}
	\item Подгруппа $\mathbb{H}$ группы $\mathbb{G}$ является нормальной, если для любого элемента $h \in \mathbb{H}$ и любого $g \in \mathbb{G}$ элемент $ghg^{-1}$ лежит в $\mathbb{H}$. 
	\item Элементы $g_1$ и $g_2$ группы $\mathbb{G}$ называются сопряженными, если существует элемент $h \in \mathbb{G}$, для которого $hg_1h^{-1}=g_2$. Класс сопряженности --- множество элементов группы $\mathbb{G}$, сопряженных заданному $g \in \mathbb{G}$.
	\end{enumerate}
	Тогда для любого элемента $h \in \mathbb{H}$ все его сопряженные элементы также лежат в $\mathbb{H}$. Значит, $\mathbb{H}$ является объединением классов сопряженности по всем $h \in \mathbb{H}$. Теперь рассмотрим класс сопряженности [$h_k$], состоящий из одного элемента:
	\begin{enumerate}
	    \item $\exists g \in \mathbb{G}:  g h_k g^{-1} = h_k$
	    \item $\forall x \in \mathbb{G} \ \nexists g \in \mathbb{G}: g h_k g^{-1} = x$ или $\forall g \in \mathbb{G}: g h_k g^{-1} = h_k$
	\end{enumerate}
	Видно, что второе условие удовлетворяет определению нормальной подгруппы и в $\bigcup$[$h_k$] будут лежать только сами эти элементы.
\end{solution}

\end{document}
