\begin{task}{11}
	На плоскости выбрано конечное количество точек, находящихся в общем положении (никакие три точки не лежат на одной прямой, никакие две не совпадают). Некоторые из выбранных точек соединены отрезками. Если два отрезка пересекаются, то их можно заменить двумя другими с концами в тех же точках (отрезки, имеющие лишь общий конец, не считаются пересекающимися). Может ли этот процесс продолжаться бесконечно? [Необходимо решить задачу непременно методом потенциалов, явно указав, какая функция используется в качестве «потенциала».]
\end{task}

\begin{solution} 
	Рассмотрим два пересекающихся отрезка. Зададим потенциал как сумму длин всех отрезков.  Четыре точки задают четырехугольник, сумма длин диагоналей в нём меньше сумм длин любых двух сторон. Тогда при замене пересекающихся отрезков на не пересекающиеся потенциал уменьшается. Так как у нас конечно число состояний системы, это значит что потенциал достигнет в какой-то момент минимума. Следовательно, процесс не может продолжаться бесконечно.
\end{solution}