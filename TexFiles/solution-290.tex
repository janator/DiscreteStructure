\begin{task}{290}
    Сколькими способами можно рассадить за круглым девятиместным столом трёх англичан, трёх французов и трёх турок, так, чтобы никакая тройка соотечественников не сидела рядом (сидящие рядом пары соотечественников допустимы)? Рассадки, совмещающиеся поворотами круглого стола, считаются одинаковыми. 
\end{task}

\begin{solution}
	Пусть места пронумерованы. Тогда число вариантов рассадить людей за столом будет 9!. Посчитаем число способов, в которых три француза  сидят рядом. Будет девять вариантов выбрать место для них, еще 3! варианта рассадить их по этим трем соседним местам. Далее остаётся шесть мест, по которым можно рассадить оставшихся людей 6! способами. В итоге получаем \(9! \cdot 3! \cdot 6!\). Для англичан и турок способов будет столько же, но рассадки в которых сидят по трое рядом и французы и турки или и французы и англичане мы посчитали несколько раз. Пусть рядом сидят французы и англичане. Для французов выбрать три места всё также девять способов, из оставшихся шести мест выбрать три подряд для англичан можно четерьмя способами. И и тех и других между собой можно разместить 3! способами. Для французов и турок и турок и англичан симметрично. Теперь посчитаем сколько способов когда все сидят по трое. Выбрать для французов места девять способов, для турок два (слева или справа от французов, и во всех группах 3! вариантов рассадки между собой. В итоге:
	\[ 9! - 9 \cdot 3 \cdot 3! \cdot 6! + 9 \cdot 3 \cdot 3! \cdot 4 \cdot 3! \cdot 3! - 9 \cdot 3! \cdot 2 \cdot 3! \cdot 3! = 265680.\]
	Вспомним что повороты стола совпадают, так что:
	\[N = \frac{265680}{9} = 29520.\]
	Ответ: 29520.
\end{solution}